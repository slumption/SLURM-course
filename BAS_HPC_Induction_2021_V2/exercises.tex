\documentclass[handout]{beamer} %No presenter effects
%\documentclass{beamer} %Presenter effects

\usepackage{pgfpages}
%\pgfpagesuselayout{4 on 1}[a4paper,border shrink=5mm,landscape] %4 sided handout

\mode<presentation>
{
  \usetheme{default}
  \usecolortheme{dove}
  \setbeamertemplate{navigation symbols}{}
  \setbeamercovered{transparent}
}

\usepackage[english]{babel}
\usepackage[latin1]{inputenc}
\usepackage{multicol}
\usepackage{graphicx}
\usepackage{tikz}
\usepackage{comment}
\usepackage{textcomp}
\usepackage{xcolor}
\usepackage{wrapfig}

\RequirePackage{inputenc}

\graphicspath{ {./imgs/} }

%\usepackage{times}
%\usepackage[T1]{fontenc}
% Or whatever. Note that the encoding and the font should match. If T1
% does not look nice, try deleting the line with the fontenc.


\title[British Antarctic Survey] % (optional, use only with long paper titles)
{An Introduction to High Performance Computing 2021}

%\subtitle
%{Presentation Subtitle} % (optional)

\author[Paul Sumption]% (optional, use only with lots of authors)
{Paul Sumption\\ \texttt{pasump@bas.ac.uk}}

%\author[Andrew England] % (optional, use only with lots of authors)
%{Andrew England\\ \texttt{aeng@bas.ac.uk}}

%{F.~Author\inst{1} \and S.~Another\inst{2}}
% - Use the \inst{?} command only if the authors have different
%   affiliation.

%{BAS HPC service (http://ictdocs.nerc-bas.ac.uk/wiki/index.php/HPC:User_Guide)}
% - Use the \inst command only if there are several affiliations.
% - Keep it simple, no one is interested in your street address.

%\date[15/01/2022] % (optional)

\subject{Courses}

\begin{document}

{
\usebackgroundtemplate{\includegraphics[width=\paperwidth]{header_and_footer}}%
\begin{frame}
\titlepage
\end{frame}
}

{
\section{Login}
\usebackgroundtemplate{\includegraphics[width=\paperwidth]{footer_1}}%
\begin{frame}{Exercise 1: Login to a BAS HPC Workstation}
\begin{itemize}
\item Login to any of the workstations: bslws01...bslws12 
\item Configure X2go (it runs on Mac, Linux and Windows) 
\item X2go will also allow us to use a graphical desktop in our later exercises
\end{itemize}
\end{frame}
}

{
\usebackgroundtemplate{\includegraphics[width=\paperwidth]{footer_3}}%
\begin{frame}{Exercise 1: X2go solution}
\text Follow the \href{https://servicedesk.bas.ac.uk/app/itdesk/ui/solutions/60624000019605299/details}{\color{blue}{service desk solution}} to setup your client
\includegraphics[scale=0.22]{servicedesk-x2go-solution.png}
\end{frame}
}


{
\usebackgroundtemplate{\includegraphics[width=\paperwidth]{footer_4}}%
\begin{frame}{Exercise 1: X2go session preferences }
\begin{minipage}{0.5\textwidth}
\begin{figure}[H]
\includegraphics[scale=0.2]{x2g0-bslws06-prefs.png}
%\caption{\label{fig:mate-session}Session preferences}
\end{figure}
\end{minipage} \hfill
\begin{minipage}{0.35\textwidth}
\begin{itemize}
\item Follow the instructions in the solution for your operating system
\item Configure a session preference for a workstation
\item My example uses ssh keys, you can just use passwords
\end{itemize}
\end{minipage}
\end{frame}
}

{
\usebackgroundtemplate{\includegraphics[width=\paperwidth]{footer_5}}%
\begin{frame}{Exercise 1: X2go session launch }
\begin{minipage}{0.5\textwidth}
\begin{figure}[H]
\includegraphics[scale=0.25]{x2go-bslws06-mate}
%\caption{\label{fig:mate-session}A MATE session}
\end{figure}
\end{minipage} \hfill
\begin{minipage}{0.35\textwidth}
\begin{itemize}
\item Click on a session and launch it
\item You should see a MATE desktop
\item {\color{red}{Demonstration}} 
\item {\color{red}{Troubleshooting}}
\end{itemize}
\end{minipage}

\end{frame}
}





{
\section{Simple command line operations}
\usebackgroundtemplate{\includegraphics[width=\paperwidth]{footer_4}}%
\begin{frame}{Exercise 2: Simple command line operations}
\begin{itemize}

\item[(a)]{List your current directory (folder) using \alert{ls -al}. Use \alert{df -h} to see the various cluster filesystems, their sizes and their current total usages. You will be on a random login node -- use \alert{hostname} to confirm which one, and \alert{w} to find out who else is using it. Use \alert{lstopo} to find out more about the internal structure of the login node.}

\item[(b)]{Examine your personal filesystem quotas with the command \alert{quota}.}
 \visible<2->{\begin{description}
   \item{You should see a 40GB quota on /home, a 1TB block and 1024k file quota on /rds-d2 (which corresponds to $\tilde{}$/rds/hpc-work).}
   \end{description}}

\end{itemize}
\end{frame}
}

{
\section{Simple command line operations}
\usebackgroundtemplate{\includegraphics[width=\paperwidth]{footer_5}}%
\begin{frame}{Exercise 2: Simple command line operations}
\begin{itemize}

\item[(a)]{List your current directory (folder) using \alert{ls -al}. Use \alert{df -h} to see the various cluster filesystems, their sizes and their current total usages. You will be on a random login node -- use \alert{hostname} to confirm which one, and \alert{w} to find out who else is using it. Use \alert{lstopo} to find out more about the internal structure of the login node.}

\item[(b)]{Examine your personal filesystem quotas with the command \alert{quota}.}
 \visible<2->{\begin{description}
   \item{You should see a 40GB quota on /home, a 1TB block and 1024k file quota on /rds-d2 (which corresponds to $\tilde{}$/rds/hpc-work).}
   \end{description}}

\end{itemize}
\end{frame}
}

{
\section{File transfer}
\usebackgroundtemplate{\includegraphics[width=\paperwidth]{footer_6}}%
\begin{frame}{Exercise 3: Laptop setup}
Before attempting exercise 3:
\begin{itemize}
\item{\url{https://www.hpc.cam.ac.uk/training-courses}}
\item{Download the file \alert{exercises.tgz} to your desktop.}
\item{You will need an sftp client on your laptop to then transfer this file to the cluster.}
\item{Mac users - open a terminal, you can use sftp from the command line}
\item{Windows users - download winsftp}
\end{itemize}
\end{frame}
}

{
\section{File transfer}
\usebackgroundtemplate{\includegraphics[width=\paperwidth]{footer_7}}%
\begin{frame}{Exercise 3: File transfer}
\begin{itemize}
\item{Use SFTP to transfer the file \alert{exercises.tgz} to your Research Computing Service training account directory $\tilde{}$/rds/hpc-work.}
\visible<2->{\begin{description}
\item[\emph{Hints:}]{\small The command is \alert{sftp}. Use the same remote host, username and password as in the previous exercise.\\\smallskip
  Use \alert{cd rds/hpc-work} to change the target directory, then \alert{put exercises.tgz} to transfer the file from your desktop to the target directory on the Research Computing Service cluster. Use \alert{quit} to close the connection.}
\end{description}Optionally, copy the file over again using \alert{rsync}.}
\end{itemize}
\end{frame}
}

{    
\section{File transfer}
\usebackgroundtemplate{\includegraphics[width=\paperwidth]{footer_8}}%
\begin{frame}{Exercise 3: File transfer (ctd)}
\begin{itemize}
\item{Switch back to the SSH session you created in the previous exercise. Verify that the file is now present by using \alert{ls}.}
\visible<2->{\begin{description}
\item[\emph{Hints:}]{Do \alert{ls -al\quad$\tilde{}$/rds/hpc-work/}. Note that you can often reduce typing by pressing \alert{TAB}.}
\end{description}}
\item{Unpack the tar archive to create an exercise subdirectory.}
\visible<3->{\begin{description}
\item[\emph{Hints:}]{Do \alert{cd\quad$\tilde{}$/rds/hpc-work/} then \alert{tar -zxvf exercises.tgz}.}
\end{description}}
\end{itemize}
\end{frame}
}    

{
\section{Modules and Compilers}
\usebackgroundtemplate{\includegraphics[width=\paperwidth]{footer_1}}%
\begin{frame}{Exercise 5: Modules and Compilers}
\begin{itemize}
\item{Go to the \alert{exercises} directory of your cluster account.}
\visible<2->{\begin{description}
\item[\emph{Hints:}]{\small Firstly you may need to review Exercise~1 in order to reconnect to your cluster account. At the remote command prompt, change to the exercises directory (\alert{cd $\tilde{}$/rds/hpc-work/exercises}).}
\end{description}}
\item{Try to compile the \alert{hello.c} program using the default \alert{gcc} compiler (it will fail because there is a deliberate bug).}
\visible<3->{\begin{description}
\item[\emph{Hints:}]{\small \alert{gcc hello.c -o hello}}
\end{description}}
\end{itemize}
\end{frame}
}

{
\section{Modules and Compilers (ctd)}
\usebackgroundtemplate{\includegraphics[width=\paperwidth]{footer_2}}%
\begin{frame}{Exercise 5: Modules and Compilers}
\begin{itemize}
\item{To fix the problem, open the \alert{hello.c} file in an editor (e.g. \alert{gedit}, \alert{nano}, \alert{emacs}).}
\visible<4->{\begin{description}
\item[\emph{Hints:}]{\small Launch gedit in the background by doing \alert{gedit\&}. A gedit window should appear. Remove the word \alert{BUG}, save the file and recompile. Do \alert{./hello} to run the program.}
\end{description}}
\end{itemize}
\end{frame}
}

{
\usebackgroundtemplate{\includegraphics[width=\paperwidth]{footer_3}}%
\begin{frame}{Exercise 5: Modules and Compilers (ctd)}
  \begin{itemize}
  \item{The default version of gcc on the RCS HPC clusters is 4.8.5. Compile hello.c again with \alert{gcc 5.4.0}.}
   \visible<2->{\begin{description}
\item[\emph{Hints:}]{\small module av, module load, then \alert{gcc hello.c -o hello2}}
\end{description}}
\item{Launch the Matlab GUI. Note this should work from either the SSH command-line or remote desktop sessions.}
\visible<3->{\begin{description}
\item[\emph{Hints:}]{\small \alert{module load matlab} then run: \alert{matlab\&}}
\end{description}}
\item{Quit Matlab and launch it again without the graphical desktop interface. This is the way to launch it inside a batch job.}
\visible<4->{\begin{description}
\item[\emph{Hints:}]{\alert{matlab -nodisplay -nojvm -nosplash}}
\end{description}}
\end{itemize}
\end{frame}
}

{
\section{Submitting Jobs}
\usebackgroundtemplate{\includegraphics[width=\paperwidth]{footer_4}}%
\begin{frame}{Exercise 6: Submitting Jobs (Matlab)}
\begin{itemize}
\item{Submit a job which will run \alert{matlab} on the \alert{file.m} command file (which contains just the Matlab \alert{ver} command).}
\visible<2->{\begin{description}
\item[\emph{Hints:}]{\scriptsize\begin{enumerate}
\item{Load the matlab module at the place indicated in the file \alert{job\_script} in your exercises directory.}
\item{Set the value of application to\hfill\break{}\alert{\"{}matlab -nodesktop -nosplash -nojvm\"{}}}
\item{Set the value of options to \alert{\"{}-r file\"{}}}
\item{Submit the job with \alert{sbatch job\_script}. The jobid is then printed.}
\item{Watch the job in the queue with \alert{squeue}.}
\item{After it has disappeared, open the output file \alert{slurm-jobid.out} in your editor. It should contain a list of licensed Matlab features from the ver command.}
  \end{enumerate}%
}
\end{description}}
\end{itemize}
\end{frame}
}

{
\usebackgroundtemplate{\includegraphics[width=\paperwidth]{footer_5}}%
\begin{frame}{Exercise 7: Submitting Jobs (serial or threaded application)}
\begin{itemize}
\item{Submit a job which will run a copy of your hello program on 1 cpu.}
\visible<2->{\begin{description}
\item[\emph{Hints:}]{\scriptsize\begin{enumerate}
\item{Edit the script \alert{job\_script} in your exercises directory. Set:\hfill\\
\alert{\#SBATCH --nodes=1}\hfill\\
\alert{\#SBATCH --ntasks=1}\hfill\\
\alert{application="./hello"}}
\item{Submit the job with \alert{sbatch job\_script}. The jobid is then printed.}
\item{Watch the job in the queue with \alert{squeue}.}
\item{After it has disappeared, open the output file \alert{slurm-jobid.out} in your editor. There should be exactly one ``Hello, World!'' message.}
\end{enumerate}%
}
\end{description}}
\end{itemize}
\end{frame}
}

{
\usebackgroundtemplate{\includegraphics[width=\paperwidth]{footer_6}}%
\begin{frame}{Exercise 7: Submitting Jobs (serial or threaded application)}
\begin{itemize}
\item Experiment with varying the number of nodes and tasks.
\item Note you will need to launch the application with \alert{srun} to actually use more than 1 cpu.
\end{itemize}
\end{frame}
}

{
\usebackgroundtemplate{\includegraphics[width=\paperwidth]{footer_7}}%
\begin{frame}{Exercise 8: Submitting Jobs (R)}
\begin{itemize}
\item{R jobs may be serial, threaded, or even MPI parallel depending on the packages used. Submit a job which will run the trivial script \alert{hello.r} program on 1 cpu.}
\visible<2->{\begin{description}
\item[\emph{Hints:}]{\scriptsize\begin{enumerate}
\item{Edit the script \alert{job\_script} in your exercises directory. Set:\hfill\\
\alert{\#SBATCH --nodes=1}\hfill\\
\alert{\#SBATCH --ntasks=1}\hfill\\
\alert{application="Rscript"}\hfill\\
\alert{options="hello.r"}}
\item{Submit the job with \alert{sbatch job\_script}. The jobid is then printed.}
\end{enumerate}%
}
\end{description}}
\item{Repeat this using a different version of R.}
\end{itemize}
\end{frame}
}

{
\section{Array Jobs}
\usebackgroundtemplate{\includegraphics[width=\paperwidth]{footer_8}}%
\begin{frame}{Exercise 9: Array Jobs}
\begin{itemize}
\item{Submit your last job in the form of an array with indices 1-64. Use -H with sbatch to mark the array as held (so that it won't run immediately).}
\visible<2->{\begin{description}
\item[\emph{Hints:}]{\scriptsize\begin{enumerate}
\item{Use \alert{sbatch -H -{}-array=1-64 job\_script}}
\item{Use \alert{squeue -u userid} to see your array job. Note that \alert{-r} reports each array element individually.}
\end{enumerate}%
}
\end{description}}
\end{itemize}
\end{frame}
}

{
\section{Array Jobs (ctd)}
\usebackgroundtemplate{\includegraphics[width=\paperwidth]{footer_1}}%
\begin{frame}{Exercise 9: Array Jobs}
\begin{itemize}
\item{Release array element 1 and allow it to run. Then release the others.}
\visible<3->{\begin{description}
\item[\emph{Hints:}]{\scriptsize\begin{enumerate}
\item{Use \alert{scontrol release \$\{{SLURM\_ARRAY\_JOB\_ID}\}\_{{\color{red}1}}}}
\item{Use \alert{squeue -u userid} again to watch what happens.}
\item{Release the others with\hfill\break
\null\qquad scontrol release \$\{{SLURM\_ARRAY\_JOB\_ID}\}\hfill\break
i.e. use the array id to release the entire array.}
  \item{When all the jobs complete you should have 64 slurm-\$\{SLURM\_ARRAY\_JOB\_ID\}\_N.out files saying hello from various cpus on possibly multiple nodes.}
\end{enumerate}%
}
\end{description}}
\end{itemize}
\end{frame}
}

\end{document}
